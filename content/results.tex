
\chapter{Constructions of the Quantum Base Size}

This chapter gives explicit computations of the \emph{quantum base size} defined in Chapter 3. We prove the number 
of queries required for exact and bounded error learning for the dihedral groups $D_{4k}$.

We work through the example of the dihedral groups in much detail to illustrate how one might go about computing 
these bounds for more families of groups.

\section{Character Table of the Dihedral Group}

The \emph{dihedral group} of order $2n$ is given by
\[
    D_{2n} := \langle a,b \mid a^n = b^2 = 1,\ bab^{-1} = a^{-1} \rangle
\]
It arises naturally as the planar symmetries of the $n$-gon. For this reason, elements of the form $a^i$ are often 
referred to as rotations and elements of the form $a^ib$ are referred to as reflections.

The character table for the case when $n$ is even is slightly different from the case when $n$ is odd. To keep 
things as clear as possible and to avoid too much repetition, we will do everything only for the case where $n$ is 
even, i.e. for the dihedral group of order $4k$ for some integer $k$. The results for the odd case readily follow 
from the exact same approach, only there are fewer conjugacy classes and thus fewer cases to check.

%\subsection{Conjugacy Classes of $D_{4k}$}

%The center of $D_8$ is given by
%\[
    %\mathbf{Z}(D_8) = \{1, a^{2k}\}
%\]
%Thus $\cl(1)$ and $\cl(a^k)$ are both of size 1. From the relation we know that for each $1 \leq i \leq k-1$  that 
%$a^i$ is conjugate to $a^{-1}$. 

%Furthermore note the rewriting the second relation gives
%\[
    %ab = ba^{-1}
%\]
%Thus





\subsection{Irreducible Characters of $D_{4k}$}

\subsubsection{Linear Characters}
The linear characters are precisely the homomorphisms $\lambda : D_{4k} \rightarrow \C^\times$. We can compute all 
such homomorphisms by looking at how they must act on the generators. 

Let $\lambda : D_{4k} \rightarrow \C^\times$ be a homomorphism. Since $b^2 = 1$ we must have $\lambda(b)^2 = 1$, 
hence the only possible values for $\lambda(b)$ are $1$ and $-1$. Since we have the relation $abab = 1$, then $1 = 
\lambda(abab) = \lambda(a)\lambda(b)\lambda(a)\lambda(b) = \lambda(a)^2$. Thus $\lambda(a)$ must also be either $1$ 
or $-1$. This leaves us with four possible choices, each of which is a linear character of $D_{4k}$

The following table gives the four linear characters of $D_{4k}$
\begin{center}
\begin{tabular}{c|ccccc}
    $g$ &$1$ & $a^k$ & $a^i$ & $b$ & $ab$ \\ \hline
    $\lambda_1$ & $1$ & $1$ & $1$ & $1$ & $1$ \\
    $\lambda_2$ & $1$ & $1$ & $1$ & $-1$ & $-1$ \\
    $\lambda_3$ & $1$ & $(-1)^k$ & $(-1)^i$ & $1$ & $-1$ \\
    $\lambda_4$ & $1$ & $(-1)^k$ & $(-1)^i$ & $-1$ & $1$ \\
\end{tabular}
\end{center}

\subsubsection{The Other Irreducible Characters}
Consider the family of representations $\{\rho_j\}$ given by parameter $1 \leq j \leq k-1$ defined by
\[
    a \mapsto \rmat{\zeta_{2z}^j}{0}{0}{\zeta_{2k}^{-j}}, \quad \ \ b \mapsto \rmat{0}{1}{1}{0}
\]

Let $\psi_j$ denote the character of the representation $\rho_j$. Taking the trace gives us
\begin{align*}
    &\psi_j(a^nb^m) = \begin{cases}
        \zeta_n^j + \zeta_n^{-j}, &\text{if } b = 0 \\
        0,                &\text{if } b =1
                    \end{cases}
\end{align*}
In terms of the conjugacy classes this is
\begin{center}
\begin{tabular}{c|ccccc}
    $g$ &$1$ & $a^k$ & $a^i$ & $b$ & $ab$ \\ \hline
    $\psi_j$ & $ \ \ 2 \ \ $ & $\ 2(-1)^j\ $ & $\zeta_{2k}^{ij} + \zeta_{2k}^{-ij}$ & $\ \ 0 \ \ $ & $ \ \ 0 \ \ $
\end{tabular}
\end{center}

In order to show $\psi_j$ is irreducible we compute $\langle \psi_j, \psi_j \rangle$.
\begin{align*}
    \langle \psi_j, \psi_j \rangle &= \frac{1}{|G|}\sum_{g \in G} \psi_j(g) \overline{\psi_j(g)} \\
                           &= \frac{1}{4k}\left[2^2 + \bigl(2(-1)^j\bigr)^2 + 
                               2\sum_{i=1}^{k-1}\left(\zeta_{2k}^{ij} + 
                               \zeta_{2k}^{-ij}\right)\left(\overline{\zeta_{2k}^{ij} + 
                       \zeta_{2k}^{-ij}}\right)\right] \\
%                             
               &= \frac{1}{4k}\left[ 4 + 4 + 2\sum_{i=1}^{k-1}\left(\zeta_{2k}^{ij} + \zeta_{2k}^{-ij}\right)^2 
               \right] \\
               &= \frac{1}{4k}(8 + 2(4)(k-1)) \\
               &= 1 \end{align*}

Thus by Theorem 2.7, $\psi_j$ is an irreducible character. 

We have found 4 irreducible linear characters and $k-1$ irreducible characters of degree 2. There are $k+3$ 
conjugacy classes and so this must be a complete set of characters. Hence the character table of $D_{4k}$ is 

\begin{center}
\begin{tabular}{c|ccccc}
    $g$ &$1$ & $a^k$ & $a^i$ & $b$ & $ab$ \\ \hline
    $\lambda_1$ & $1$ & $1$ & $1$ & $1$ & $1$ \\
    $\lambda_2$ & $1$ & $1$ & $1$ & $-1$ & $-1$ \\
    $\lambda_3$ & $1$ & $(-1)^k$ & $(-1)^i$ & $1$ & $-1$ \\
    $\lambda_4$ & $1$ & $(-1)^k$ & $(-1)^i$ & $-1$ & $1$ \\
    $\psi_j$ & $2$ & $2(-1)^j$ & $\zeta_{2k}^{ij} + \zeta_{2k}^{-ij}$ & $0$ & $0$
\end{tabular}
\end{center}

\section{Quantum Base Size of the Dihedral Group}




\subsection{Permutation Character}
Let $G = D_{4k}$. Consider the natural action of $D_{4k}$ on the vertices of the $2k$-gon. We want to compute 
$\text{Fix}(g)$ for each conjugacy class.


Every rotation (excluding the identity) will fix no points. Similarly an odd number of rotations and then a flip 
will never yield any fixed points.  The only elements that will fix anything will be the identity (which will fix 
all $2k$ vertices) and the primitive reflections, which will each fix exactly the two vertices that lie in the axis 
of reflection. Thus the permutation character $\pi(g) = \text{fix}(g)$ is given by
\begin{center}
\begin{tabular}{c|ccccc}
    $g$ &$1$ & $a^k$ & $a^i$ & $b$ & $ab$ \\ \hline
    $\pi(g)$ & $2k$ & $0$ & $0$ & $2$ & $0$ \\
\end{tabular}
\end{center}
for all $1 \leq i \leq k-1$.

\subsection{Queries for Exact Learning}

We want to determine $\gamma(D_{4k})$. In order to do this, we look at the irreducible constituents of the tensor 
powers of the permutation character. We need to find the smallest integer $t$ such that $\pi^t$, the $t$th tensor 
power of pi, contains every irreducible character as a constituent. That is, the smallest $t$ such that $\langle 
\pi^t, \chi \rangle \neq 0$ for all irreducible characters $\chi$.


We summarize the information we have computed so far about the characters of $D_{4k}$

\def\cl{\text{cl}}
\begin{center}
\begin{tabular}{c|ccccc}
    $g$ &$1$ & $a^k$ & $a^i$ & $b$ & $ab$ \\ \hline
    $|\cl(g)|$ & $1$ & $1$ & $2$ & $k$ & $k$ \\ \hline
%
    $\lambda_1$ & $1$ & $1$ & $1$ & $1$ & $1$ \\
    $\lambda_2$ & $1$ & $1$ & $1$ & $-1$ & $-1$ \\
    $\lambda_3$ & $1$ & $(-1)^k$ & $(-1)^i$ & $1$ & $-1$ \\
    $\lambda_4$ & $1$ & $(-1)^k$ & $(-1)^i$ & $-1$ & $1$ \\
    $\psi_j$ & $2$ & $2(-1)^j$ & $\zeta_{2k}^{ij} + \zeta_{2k}^{-ij}$ & $0$ & $0$ \\ \hline
    $\pi$ & $2k$ & $0$ & $0$ & $2$ & $0$ 
\end{tabular}
\end{center}
where $\psi_j$ is a family of representations, one for each $1 \leq j \leq k-1$ and $a^i$ is a unique conjugacy 
classes for each $1 \leq i \leq k-1$.

To determine which irreducible characters are constituents of $\pi$, we compute $\langle \pi, \chi\rangle$ for each 
irreducible $\chi$.  Since $\pi$ is nonzero only on the classes $1$ and $b$ we omit these terms.
\begin{align*}
    \langle \pi, \lambda_1 \rangle &= \frac{1}{|G|}\bigg( (1)(1)(2k) + (2)(1)(k) \bigg)  = \frac{1}{4k}(2k + 2k) = 
    1 \\
%
    \langle \pi, \lambda_2 \rangle &= \frac{1}{|G|}\bigg( (1)(1)(2k) + (2)(-1)(k) \bigg)  = \frac{1}{4k}(2k - 2k) = 
    0 \\
%
    \langle \pi, \lambda_3 \rangle &= \frac{1}{|G|}\bigg( (1)(1)(2k) + (2)(1)(k) \bigg)  = \frac{1}{4k}(2k - 2k) = 
    1 \\
%
    \langle \pi, \lambda_4 \rangle &= \frac{1}{|G|}\bigg( (1)(1)(2k) + (2)(-1)(k) \bigg)  = \frac{1}{4k}(2k - 2k) = 
    0 \\
    \langle \pi, \psi_j \rangle &= \frac{1}{2n}(2n) = 1 
\end{align*}

Thus, the decomposition of $\pi$ into irreducibles is given by
\[
    \pi = \lambda_1 + \lambda_3 + \sum_{j=1}^{k-1} \psi_j
\]
Hence, the irreducibles $\lambda_2$ and $\lambda_4$ are not constituents of $\pi$ and so $\gamma(D_{4k})$ must be 
greater than one. By Theorem 2.8, every irreducible must appear in the decomposition of $\pi^t$ for some $t \leq 
2$, thus we know without any further computation that $\gamma(D_{4k}) = 2$.

\subsection{Queries for bounded probability of learning}
Although this means a quantum oracle does not help you solve the problem exactly with fewer queries, it is better 
when you allow for finding a solution with high probability.

As given in chapter 3, the number of queries needed to succeed with probability greater than 2/3 is the smallest 
positive integer $t$ such that
\[
    \frac{1}{|G|} \sum_{\chi \in N^{\otimes t}} \chi(e)^2  \ \geq\  \frac{2}{3}
\]

We can find this value using what we computed in section 4.2.2.  The constituent irreducibles of $\pi$ are 
$\lambda_1, \lambda_3,$ and $\psi_j$ for each $1 \leq j \leq k-1$. These are 2 characters of degree 1 and $k-1$ 
characters of degree 2. Thus
\[
    \frac{1}{|D_{4k}|} \sum_{\chi \in N^{\otimes t}} \chi(e)^2 = \frac{1}{4k}\bigl( 2(1^2) + (k-1)(2^2) \bigr) = 
    \frac{1}{4k}(2 + 4k - 4) = 1 - \frac{1}{2k}
\]

Therefore one can succeed using only one query with probability $1 - \frac{1}{2k}$, which is bounded above 2/3 for 
every dihedral group of order $4k$.

%\section{The General Linear Group $GL(2,q)$}

%Here we present the analogous results for $\GL(2, q)$, the group of $2 \times 2$ invertible matrices over the 
%finite field with $q$ elements. 

%\subsection{Conjugacy Classes}

%The conjugacy classes of $\GL(2,q)$ can be classified into four families.
%\begin{enumerate}
    %\item Scalar multiples of the identity i.e. matrices of the form
        %\[
           %aI =  \rmat{a}{0}{0}{a}
        %\]
        %for some $a \in \F_q$. Each $aI$ is in the center of $\GL(2,q)$ and so $|\cl(aI)| = 1$ for all $a \in 
        %\F_q$.

    %\item Upper triangle matrices. These are the matrices of the form
    %\[
        %u_a = \rmat{a}{1}{0}{a}
    %\]
    %for some $a \in F_q$. 

%\item Diagonal matrices.  These are of the form
    %\[
        %d_{a,b} = \rmat{a}{0}{0}{b}
    %\]
%\item Matrices with eigenvalues not in the field
    %\[
        %\rmat{0}{1}{-a^{q+1}}{a+a^q}
    %\]
    %where $a$ is in $\F_{q^2}$ but not in $\F_q$.
%\end{enumerate}


%\begin{center}
%\begin{tabular}{r|cccc}
    %Family & \vphantom{\Bigg|} $aI$ & $u_a$ & $d_{a,b}$ & $v_a$  \\
%Parameters & \vphantom{\Bigg|} $a \in \F_q$ & $a \in \F_q$ & $a,b \in \F_q \setminus{0}, a\neq b$ & $a \in \F_q$  
%\\
    %\vphantom{\Bigg|}Representatives & $ \rmat{a}{0}{0}{a}$ & $ \rmat{a}{1}{0}{a}$ & $ \rmat{a}{0}{0}{b} $ & $ 
    %\rmat{0}{1}{-a^{q+1}}{a + a^q}$ \\
%\# Classes in Family & \vphantom{\Bigg|}$q-1$ & $q-1$ & $\frac{1}{2}(q-1)(q-2)$ & $\frac{1}{2}(q^2 -q)$ \\
%\vphantom{\Bigg|} Size of Each Class & 1 & $q^2 - 1$ & $q(q+1)$ & $q(q-1)$
%\end{tabular}
%\end{center}

%\subsection{Character Table}

%The character table of $\GL(2,q)$ is given by

%\begin{center}
%\begin{tabular}{r|cccc}
    %$g$ & $\vphantom{\Bigg|}$ $aI$ & $u_a$ & $d_{a,b}$ & $v_a$  \\ \hline
    %$\lambda_i$ & \vphantom{\Bigg|} $a \in \F_q$ & $a \in \F_q$ & $a,b \in \F_q \setminus{0}, a\neq b$ & $a \in $
    %$\psi_i$ & \vphantom{\Bigg|} $a \in \F_q$ & $a \in \F_q$ & $a,b \in \F_q \setminus{0}, a\neq b$ & $a \in $
    %$\varphi_{i,j}$ & \vphantom{\Bigg|} $a \in \F_q$ & $a \in \F_q$ & $a,b \in \F_q \setminus{0}, a\neq b$ & $a \in 
    %$\chi_i$ & \vphantom{\Bigg|} $a \in \F_q$ & $a \in \F_q$ & $a,b \in \F_q \setminus{0}, a\neq b$ & $a \in $
%\# Classes in Family & \vphantom{\Bigg|}$q-1$ & $q-1$ & $\frac{1}{2}(q-1)(q-2)$ & $\frac{1}{2}(q^2 -q)$ \\
%\vphantom{\Bigg|} Size of Each Class & 1 & $q^2 - 1$ & $q(q+1)$ & $q(q-1)$
%\end{tabular}
%\end{center}


%\section{Query Complexity for the General Linear Group}
%Classically, the query complexity is 2. Any single query will only give at best a column or equivalent amount of 
%information.

%Let $\pi$ be the permutation character of the action of $\GL(2, q)$ on $(\F_q)^2$ by left multiplication. Consider 
%an arbitrary element $g \in GL(2,q)$. It is of the form
%\[
    %g = \rmat{a}{b}{c}{d}
%\]
%for some $a,b,c,d \in \F_q$ such that $ab \neq cd$. The action of $g$ on a vector $(x,y) \in (\F_q)^2$.
%\[
    %g\cdot(x,y) = \rmat{a}{b}{c}{d} \rvec{x}{y} = \rvec{ax + by}{cx + dy}
%\]
%Checking this action on the conjugacy classes as given before we can compute the value of the permutation character 
%to be given by

%\begin{center}
%\begin{tabular}{c|cccc}
    %\vphantom{\Bigg|} $g$ & $ \bmat{1}{0}{0}{1}$ & $\bmat{1}{1}{0}{1}$ &
    %$\bmat{1}{0}{0}{b}$ & Else   \\ \hline
    %\vphantom{\Bigg|} $|\cl(g)|$ & $1$ & $1$ & $q-2$ & \ \\
    %\vphantom{\Bigg|} $\pi(g)$ & $q^2$ & $q$ & $q$ & $1$
%\end{tabular}
%\end{center}

%We can simplify the computation by noticing that $\pi -1$ is a character. In particular, it is the character 
%arising from the same action by left multiplication but instead on $(\F_q)^2 \setminus \{(0,0\}$. It has character 
%table given by
%\begin{center}
%\begin{tabular}{c|cccc}
    %\vphantom{\Bigg|} $g$ & $ \bmat{1}{0}{0}{1}$ & $\bmat{1}{1}{0}{1}$ &
    %$\bmat{1}{0}{0}{b}$ & Else   \\ \hline
    %\vphantom{\Bigg|} $|\cl(g)|$ & $1$ & $1$ & $q-2$ & \ \\
    %\vphantom{\Bigg|} $(\pi-1)$ & $q^2 - 1$ & $q -1$ & $q -1 $ & $0$
%\end{tabular}
%\end{center}
%Now let $\chi$ be an irreducible character of $G$. Then 

%\[
    %\langle \pi, \chi \rangle = \langle \pi -1, \chi \rangle + \langle 1, \chi \rangle
%\]
%where 1 is the character of the trivial representation. Since $1$ is an irreducible representation $\langle 1, \chi 
%\rangle = 0$ as long as $\chi \neq 1$. Thus it is sufficient for us to compute $\pi -1$.

%Then we compute $\langle \pi -1, \chi \rangle$ for each representation.
%\begin{align*}
    %\langle \pi - 1 ,  \lambda_i \rangle = \sum  & 
    
    
    %\end{align*}

%\begin{center}
%\begin{tabular}{c|cccc}
    %\vphantom{\Bigg|} $g$ & $ \bmat{1}{0}{0}{1}$ & $\bmat{1}{1}{0}{1}$ &
    %$\bmat{1}{0}{0}{b}$ \\ \hline
%\vphantom{\Bigg|} $|\cl(g)|$ & $1$ & $1$ & $q-2$ \\
%\vphantom{\Bigg|} $\pi(g)$ & $q^2$ & $q$ & $q$
%\end{tabular}
%\end{center}





%\subsubsection{Permutation Character}




%\begin{itemize}
    %\item Permutation Character
    %\item Big Character Table
    %\item pi -1
    %\item THere's a lot more to do here.
%\end{itemize}



%# Move to chapter 3
%For a fixed group, this process is purely mechanical. For each irreducible character $\chi$, we compute the 
%projection of $\pi$ onto $\lambda$ by taking the inner-product $\langle \pi, \lambda \rangle$. If this 
%inner-product is zero for any $\chi$, then it is not a constituent of $\pi$. We repeat this process for each 
 


