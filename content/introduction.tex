\chapter*{Introduction}
\addcontentsline{toc}{chapter}{Introduction}
\chaptermark{Introduction}
\markboth{Introduction}{Introduction}


In 1994, Peter Shor showed that if a working quantum computer could be built, it would be capable of factoring 
integers exponentially faster than is thought to be possible on a classical computer. This was perhaps the most 
prominent of a series of successes that publicly established the field of quantum computing as something worth 
seriously researching. It is a very unique subject today, as it is one of the few that is truly interdisciplinary.  
Influential papers in the field are routinely published by researchers in mathematics, physics, computer science 
and even occasionally chemistry departments. 

The primary goal of this thesis is to introduce a reader familiar with linear algebra and abstract algebra to an 
interesting open problem at the intersection quantum computing of representation theory. The first two chapters 
seek to give a self-contained introduction to quantum computing and representation theory respectively. The section 
on quantum computing is intended slightly towards someone with a mathematical background rather than a physics or 
computer science one. Finally, chapter three introduces the problem and the paper that motivates it and in chapter 
four explicit examples are computed.
    

