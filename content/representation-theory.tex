
\chapter{Representation Theory of Finite Groups}

\section{Intro}


Representation theory is a very powerful tool for analyzing groups. Representation theoretic tools give much more 
profound insight into much of group theory that can be hard to find without it. The idea of representation theory, 
is to study groups by looking at how they act on vector spaces. This allows us to use the tools of linear algebra 
to study groups. In particular, under the right conditions, we will see that quantum oracles naturally give us 
unitary representations of their underlying groups.

The purpose of this chapter will be to introduce the terminology and results from representation theory that we 
will need to investigate symmetric oracle problems in chapters 3 and 4. A reader familiar with the basic results of 
the representation theory of finite groups can skip ahead to chapter 3. 

This treatment of representation theory largely follows \cite{James&Liebeck}. I have omitted many proofs here and 
instead leave the reader to refer to \cite{James&Liebeck}

\section{Representations}

In this thesis, we will assume that all representations are over $\C $ if not explicitly stated.

\begin{definition}
    A \emph{representation} of a group $G$ over $\C$ is a homomorphism $\rho$ from $G$ to $\GL(n,F)$, for some $n$. 
    The degree of $\rho$ is the integer $n$.

    We say that $\rho$ is a \emph{faithful} representation if $G \cong \rho(G)$.
\end{definition}

\begin{definition}
    Let $G$ be a group and $V$ be a vector space over a field $F$. A \emph{representation} of $G$ is a homomorphism 
    $\rho : G \rightarrow \Aut(V)$, where $\Aut(V)$ denotes the group of linear maps from $V$ to itself. Although, 
    the representation is defined by the homomorphism $\rho$, it is common instead to refer to it by the space its 
    acting on.

    We can equivalently characterize a representation as an action of a group $G$ on a vector space $V$. In this 
    case, the action is required to be compatible with the vector space structure on $V$. That is, in addition to 
    the requirements for an action of a group on a set, we must have
\[g \cdot (\alpha v_1 + v_2) = g\cdot \alpha(v_1) + g\cdot(v_2)\]
    for all $g \in G,\ v_1, v_2 \in V,$ and $\alpha \in F$.
    We denote the action of $g \in G$ on a vector $v$ as $g \cdot v$ or $gv$.
\end{definition}

For the rest of this thesis, we will assume that all representations are over $\C$.

 
\begin{definition}
    Let $\rho$ be a representation of a group $G$ on a vector space $V$. A \emph{subrepresentation} is a vector 
    subspace $W \subset V$ that is invariant under the action of $G$.\footnotemark That is, for all $g\cdot w \in 
    W$ for all $g \in G$, $v \in W$.
    \footnotetext{This is a slight abuse of notation. The subrepresentation in this case is actually function given 
    by the homomorphism $\varphi : G \rightarrow \Aut(V)$ defined by $\varphi(g) = \rho|_W.$  }
    
    An representation is said to be irreducible if it has no nontrivial subrepresentations (the only $G$-invariant 
    subspaces of $V$ are $\{ 0 \}$ and $V$ itself).
\end{definition}



\begin{theorem}[Maschke's Theorem]
    Let $G$ be a finite group and $V$ a representation of $G$ over $\C$. If $W$ is a subrepresentation of $V$, then 
there is a subrepresentation $W'$ such that $V = W \oplus W'$.  
\end{theorem}

\begin{proof}
    Define $\pi : V \rightarrow W$ to be the projection of $V$ onto $W$. TODO rest of proof
\end{proof}



In linear algebra, when given a linear map, we commonly looked for the subspaces it was invariant on. These spaces 
were so important that we gave them their own name, \emph{eigenspaces}

\begin{definition}
    Subrepresentation

    Let $G$ be a group and $\rho : G \rightarrow \GL(n, \C)$ be a representation. 

    I've maybe messed all of this up, by not giving a formal enough treatment of it. I'm not exactly sure what to 
    do.

\end{definition}


Although, on the surface level, faithful representations might seem like what we want, we really care about 
irreducible representations.



\begin{definition}
    A representation $\rho : G \rightarrow \GL(n, \C)$ is \emph{irreducible} if its only subrepresentations are 
    $\{0\}$ and $V$. \textcolor{blue}{This and the definition I give for subprepresentation should be cleaned up 
    and made more coherent. Should I just do this in terms of FG-modules?}
\end{definition}



\begin{definition}
    Equivalent representations? I don't know if I need this. At least as a remark.
\end{definition}


\begin{theorem}[Maschke's Theorem]
    Every 
\end{theorem}

\begin{theorem}[Schur's Lemma]
    Let $V$ and $W$ be irreducible \CG-modules.

    \begin{enumerate}
        \item If $\varphi : V \rightarrow W$ is a \CG-homomorphism, then either $\varphi$ is a \CG-isomorphism, or 
            $v\varphi = 0$ for all $v \in V$. 
            
        \item If $\varphi V \rightarrow V$ is a \CG-isomorphism, then $\varphi$ is a scalar multiple of the 
            identity endomorphism $1_v$.
    \end{enumerate}
\end{theorem}

\begin{proof}
    I'll almost certainly include this one.
\end{proof}


\section{Character Theory}

One of the most surprising tools in representation theory is \emph{characters}. It's a pain to have to associate an 
entire matrix to each element of our group. What would happen if we just wrote down the trace of each matrix 
instead?




\begin{definition}
    Let $G$ be a group and $\rho : G \rightarrow \GL(n, \C)$ a representation of $G$. 

    The \emph{character of }$\rho$ is the function $\chi : G \rightarrow \C$ defined for all $g \in G$ by
    \[
        \chi(g) = \text{Tr}\,\rho(G)
    \]
\end{definition}

Recall from linear algebra that the trace can be equivalently defined as the sum of the eigenvalues of a matrix. 
This result makes it clear that the trace, and thus characters, are invariant under change of basis. This is the 
key part of the remarkable properties that the trace seems to have, one might think of it as, the simplest 
invariant of linear maps (depending on how simple you find its cousin, the determinant to be).


Much in the same way that we think of the relationship between matrices and linear maps to be defined `up to change 
of basis', we might think that groups themselves are also only defined up to inner automorphism.

Although character theory is just a shadow of the representation theory above it, the characters of a group carry a 
surprising amount of information about it. In fact, understanding the character theory of a group, is one of the 
best ways to study the representation theory of a group.

All of this prose was just written in like 30 seconds. I have no actual idea if any of this is readable or at all 
true. Anyways, back to the math.



\begin{definition}
    We say that $\chi$ is a character of a group $G$ if it is the character of some representation of $G$.
    Moreover, if $\chi$ is the character of an irreducible representation of $G$, we say that $\chi$ is an
    \emph{irreducible character of} $G$.
\end{definition}


\begin{definition}
    degree of a character??
\end{definition}

\subsection{Properties of Characters}

\begin{theorem}
    Let $\chi$ be a character of a group $G$. Suppose $g \in G$ and $g$ has order $m$. Then
    \begin{enumerate}
        \item $\chi(1) = \dim V$.
        \item $\chi(g)$ is a sum of $m$th roots of unity.
        \item $\chi(g^{-1}) = \overline{\chi(g)}$.
        \item $\chi(g)$ is a real number if $g$ is conjugate  to $g^{-1}$
    \end{enumerate}
\end{theorem}




\subsection{Character Tables}
    
    Writing the information of the characters of a group into a table, gives us an extremely compact way write down 
    the important information associated to a group.
    

\begin{definition}
    A character table of a group $G$ is of the form 
    

    [insert exmaple table]
\end{definition}




\subsection{Important Properties}


\begin{theorem}
    The number of irreducible characters of $G$ is equal to the number of conjugacy classes of $G$.
\end{theorem}




There is an inner product on the space of functions from $G$ to $\C$.
\begin{definition}
    Let $\varphi$, $\theta$ be functions from $G$ to $\C$. Define
    \[
        \langle \varphi, \theta \rangle := \frac{1}{|G|} \sum_{g \in G} \varphi(g)\overline{\psi(g)}
    \]
\end{definition}



\begin{theorem}
    Let $\chi$ and $\psi$ be inequivalent, irreducible characters of a group $G$.
    Then $\langle \chi, \chi \rangle = 1$ and $\langle \chi, \psi \rangle = 0$. 
\end{theorem}


\begin{theorem}
    Let $V$ be a \CG-module with characters $\chi$ and $\psi$, respectively. Then $V$ is irreducible if and only if 
    $\langle \psi, \psi \rangle = 1$ .
\end{theorem}


This is the most important one I think.
\begin{theorem}
    The set of irreducible characters form an orthonormal basis of the space of class functions.
\end{theorem}




















