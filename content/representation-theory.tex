
\chapter{Representation Theory of Finite Groups}

\section{Introduction}


%Representation theory is a very powerful tool for analyzing groups. Representation theoretic tools give much more 
%profound insight into much of group theory that can be hard to find without it. The idea of representation theory, 
%is to study groups by looking at how they act on vector spaces. This allows us to use the tools of linear algebra 
%to study groups. In particular, under the right conditions, we will see that quantum oracles naturally give us 
%unitary representations of their underlying groups.

The purpose of this chapter will be to introduce the terminology and results from representation theory that we 
will need to investigate symmetric oracle problems in chapters 3 and 4. A reader familiar with the basic results of 
the representation theory of finite groups can skip ahead to chapter 3. 

This treatment of representation theory largely follows \cite{James&Liebeck}. I have omitted many proofs here and 
instead leave the reader to refer to \cite{James&Liebeck}

Although there are many analogues of the results given here for fields other than $\C$ and for infinite groups, we 
will not express them in their full generality. The representations that will arise in our study of quantum oracle 
problems will be of finite groups and over $\C$. Moreover, many proofs in representation theory are substantially 
simpler and more elegant when the underlying field is restricted to $\C$.

\section{Representations}

A representation of a group $G$ is an encoding of $G$ as a set of linear maps on some vector space $V$. When a 
basis of $V$ is specified, this yields an explicit representation of each $g \in G$ as a matrix. Looking at the 
possible ways of representing $G$ as a matrix group uncovers much information about the structure of $G$ that may 
have been hidden otherwise. In particular, any encoding of $G$ into a set of matrices will be unique only up to a 
change of basis. This exposes a deep connection between the symmetry arising in vector spaces by change of basis 
and symmetry inherent in the structure of groups, arising from conjugation.


\begin{definition}
    Let $G$ be a group and $V$ be a vector space over a field $F$. A \emph{representation} of $G$ is a homomorphism 
    $\rho : G \rightarrow \Aut(V)$, where $\Aut(V)$ denotes the group of linear maps from $V$ to itself. Although, 
    the representation is defined by the homomorphism $\rho$, it is common instead to refer to it by the vector 
    space it is acting on.
    
    Equivalently, a representation can be characterized as an action of a group $G$ on a vector space $V$. In this 
    case, the action is required to be compatible with the vector space structure on $V$. That is, in addition to 
    the requirements for an action of a group on a set, we must have
\[g \cdot (\alpha v_1 + v_2) = g\cdot \alpha(v_1) + g\cdot(v_2)\]
    for all $g \in G,\ v_1, v_2 \in V,$ and $\alpha \in F$.
    We denote the action of $g \in G$ on a vector $v$ as $g \cdot v$ or $gv$.
\end{definition}


    There is yet a third way to express the information of a representation. Although this is the most abstract, it 
    provides the most concrete language for working with representations.
    
    First we must define an object called the \emph{group algebra} that is naturally associated to every group.
    \footnote{The terms \emph{group algebra} and \emph{group ring} are often used synonymously.} One can view the 
    group algebra as an encoding of a group into the structure of polynomial arithmetic.


\begin{definition}
    Let $G$ be a finite group of order $n$. The \emph{group algebra} of $G$, denoted $\C[G]$ or \CG, is defined as 
    a polynomial ring over $\C$ with one indeterminant for every $g \in G$. Thus every element of $\C[G]$ is of the 
    form
    \[
        \alpha_1 g_1 + \alpha_2 g_2 + ... + \alpha_n g_n
    \]
    where $\alpha_1, ..., \alpha_n \in \C$ and $\{g_1, ..., g_n\} = G$.
    
    The operations of addition, multiplication, and scalar-multiplication are inherited from the properties of 
    polynomial rings, except we define the multiplication of monomials to follow the group structure.
    For any $\alpha_1, \alpha_2 \in \C$ and $g_1, g_2 \in G$ we define
    \[
        (\alpha_1g_1)(\alpha_2g_2) = (\alpha_1\alpha_2)(g_1g_2)
    \]
    where $\alpha_1\alpha_2$ is the product of $\alpha_1, \alpha_2$ in $\C$ and $g_1 g_2$ is the product of $g_1, 
    g_2$ in $G$. In particular, this means $\C[G]$ is a commutative ring if and only if $G$ is abelian.
\end{definition}


A representation $\rho : G \rightarrow \Aut(V)$ defines a $\C G$-module structure on $V$ by the action
\[
    (\alpha_1g_1 + ... \alpha_ng_n) \dot v := \alpha_1\rho(g_1)v + ...+ \alpha_n\rho(g_n)v
\]
Conversely, a $\C G$-module with action $\varphi : \C G \times V \rightarrow V$ determines a representation $\rho$ 
defined by $\rho(g) = \varphi(1g)$. This establishes a correspondence between representations of $G$ and 
\CG-modules.


\section{The structure of representations}


\begin{definition}
    Let $\rho$ be a representation of a group $G$ on a vector space $V$. A \emph{subrepresentation} is a vector 
    subspace $W \subset V$ that is invariant under the action of $G$. That is, for all $g\cdot w \in W$ for all $g 
    \in G$, $v \in W$. 
    
    A representation is said to be irreducible if it has no nontrivial subrepresentations (the only $G$-invariant 
    subspaces of $V$ are $\{ 0 \}$ and $V$ itself).
\end{definition}


The next two results are fundamental to the study of representations.


\begin{theorem}[Maschke's Theorem]
    Let $G$ be a finite group and $V$ a representation of $G$ over $\C$. If $W$ is a subrepresentation of $V$, then 
there is a subrepresentation $W'$ such that $V = W \oplus W'$.  \end{theorem}


\begin{proof}
    We are given that $W$ is a subrepresentation of $V$. Let $\pi : V \rightarrow W$ be a projection of $V$ onto 
    $W$. Although $\ker \pi \oplus W = V$, we do not know that $\ker \pi$ must be a subrepresentation. In fact, it 
    will not be a subrepresentation in general. We will need to modify it slightly.

    We construct a new function $\pi_G : V \rightarrow W$ by averaging over $G$. Define it by
    \[
        \pi_G(v) = \frac{1}{|G|}\sum_{g \in G} g \cdot \pi(v) \cdot g^{-1}
    \]
\end{proof}


\begin{theorem}[Schur's Lemma]
    If M and N are two finite-dimensional irreducible representations of a group $G$, and $\varphi:M \to N$ is a 
    linear transformation that commutes with the action of the group, then $\varphi$ is either an isomorphism or 
    the $0$ map.
\end{theorem}

%\begin{theorem}[Schur's Lemma]
    %Let $V$ and $W$ be irreducible \CG-modules.

    %\begin{enumerate}
        %\item If $\varphi : V \rightarrow W$ is a \CG-homomorphism, then either $\varphi$ is a \CG-isomorphism, or 
            %$v\varphi = 0$ for all $v \in V$. 
            
        %\item If $\varphi V \rightarrow V$ is a \CG-isomorphism, then $\varphi$ is a scalar multiple of the 
            %identity endomorphism $1_v$.
    %\end{enumerate}
%\end{theorem}

%\begin{proof}
    %I'll almost certainly include this one.
%\end{proof}


\section{Character Theory}





%Recall from linear algebra that the trace can be equivalently defined as the sum of the eigenvalues of a matrix. 
%This result makes it clear that the trace, and thus characters, are invariant under change of basis. This is the 
%key part of the remarkable properties that the trace seems to have, one might think of it as, the simplest 
%invariant of linear maps (depending on how simple you find its cousin, the determinant to be).


%Much in the same way that we think of the relationship between matrices and linear maps to be defined `up to change 
%of basis', we might think that groups themselves are also only defined up to inner automorphism.

%Although character theory is just a shadow of the representation theory above it, the characters of a group carry a 
%surprising amount of information about it. In fact, understanding the character theory of a group, is one of the 
%best ways to study the representation theory of a group.

%All of this prose was just written in like 30 seconds. I have no actual idea if any of this is readable or at all 
%true. Anyways, back to the math.

\begin{definition}
    Let $\rho : G \rightarrow V$ be a representation. The map $\chi : G \rightarrow \C$ given by $\chi(g) = 
    \text{Tr}\,\rho(g)$ is called the \emph{character} of $V$.
\end{definition}

\begin{definition}
    Let $G$ be a group. We say that $\chi$ is a character of $G$ if it is the character of some representation of 
    $G$.

    Moreover, if $\chi$ is the character of an irreducible representation of $G$, we say that $\chi$ is an
    \emph{irreducible character of} $G$.
\end{definition}



We give a few basic properties of characters.
\begin{theorem}
    Let $\chi$ be a character of a group $G$. Suppose $g \in G$ and $g$ has order $m$. Then
    \begin{enumerate}
        \item $\chi(1) = \dim V$.
        \item $\chi(g)$ is a sum of $m$th roots of unity.
        \item $\chi(g^{-1}) = \overline{\chi(g)}$.
        \item $\chi(g)$ is a real number if $g$ is conjugate  to $g^{-1}$
    \end{enumerate}
\end{theorem}




\subsection{Character Tables}
    


\begin{theorem}
    The number of irreducible characters of $G$ is equal to the number of conjugacy classes of $G$.
\end{theorem}

\begin{theorem}
    The characters of a group $G$ are constant on the conjugacy classes of $G$. That is, for all $g,h \in G$, 
    $\varphi(g) = \varphi(h)$ if and only if $h = kgk^{-1}$ for some $k \in G$.
\end{theorem}

%I can give short proofs of these two.

    Writing the information of the characters of a group into a table, gives us an extremely compact way write down 
    the important information associated to a group.
    

\begin{definition}
    A character table of a group $G$ is of the form 
    

\end{definition}



\subsection{Properties of Irreducible Characters}

\begin{definition}
    A \emph{class function} of a group $G$ is a function $\varphi : G \rightarrow \C$ that is constant on the 
    conjugacy classes of $G$. As we have already seen, the characters of a group are always class functions.
    
    The set of class functions on a group $G$ form a vector space under pointwise addition. This space has an inner 
    product given by
    \[
        \langle \varphi, \psi \rangle := \frac{1}{|G|} \sum_{g \in G} \varphi(g)\overline{\psi(g)}
    \]
\end{definition}


\begin{theorem}
    The set of irreducible characters of a group $G$ form an orthonormal basis for the space of class functions on 
    $G$.
\end{theorem}

\begin{theorem}
    Let $V$ be a representation of a group $G$ and let $\psi$ be its associated character.  $V$ is irreducible if 
    and only if and only if $\langle \psi, \psi \rangle = 1$ .
\end{theorem}

\begin{proof}
    For now, just refer to James and Liebeck.
\end{proof}




\subsection{More Results on Characters}

\begin{theorem}
    Let $\chi_1, ..., \chi_n$ be the irreducible characters of a group $G$. If $\psi$ is any character of $G$, then 
    \[
        \psi = d_1\chi_1 + ... + d_k\chi_k.
    \]
    for some non-negative integers $d_1, ..., d_k$. Moreover, these coefficients count the number of times each 
    irreducible character appears:
    \[
        d_i = \langle \psi, \chi_i \rangle
    \]
    and
    \[
        \langle \psi, \psi \rangle = \sum_{i=1}^k d_i^2
    \]
\end{theorem}

This is redundant with orthonormal basis theorem, but I should introduce the combinatorial interpretation.

\begin{definition}
    Suppose that $\psi$ is a character of $G$, and that $\chi$ is an irreducible character of $G$. We say that 
    $\chi$ is a \emph{constituent} character of $\psi$ if $\langle \psi, \chi \rangle \neq 0$. Thus, the 
    constituents of $\psi$ are the irreducible characters $\chi_i$ of $G$ for which the integer $d_i$ in the 
    expression $\psi = d_1\chi_1 + ... + d_k\chi_k$ is nonzero.
\end{definition}





\subsection{Products of Characters}

\begin{definition}
    Tensor product of representations
\end{definition}

\begin{definition}
    Let $G$ be a group. Define a product on the space of class functions of $G$ by pointwise multiplication. i.e.
    for all class functions $\chi, \psi$ of $G$, 
    \[
        \chi\psi(g) = \chi(g)\psi(g) \text{ for all } g \in G
    \]
\end{definition}


\begin{proposition}
    Let $G$ be a group. Let $V$ and $W$ be representations of with characters $\chi$ and $\psi$ respectively. Then 
    the character of the representation $V \otimes W$ is the product character $\chi\psi$
\end{proposition}


\begin{theorem}\label{thm:Burnside_Brauer}(Burnside-Brauer)
    Let $\chi$ be a faithful character of $G$, and suppose that $\chi(g)$ takes precisely $r$ different values as 
    $g$ varies over all the elements of $G$. Then every irreducible character of $G$ is a constituent of one of the 
    powers $\chi^0, \chi^1, ... , \chi^{r-1}$
\end{theorem}

\begin{proof}
    TODO: Prove this one. The proof is not hard and it is very relevant.
\end{proof}




\subsection{If there is time}


Frobenius Reciprocity and Induced Subgroups
