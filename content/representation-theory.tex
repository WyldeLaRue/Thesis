
\chapter{Representation Theory of Finite Groups}




Representation theory is a very powerful tool for analyzing groups. Representation theoretic tools give much more 
profound insight into much of group theory that can be hard to find without it.



In order to study more sophisticated quantum oracle problems, we will need more sophisticated algebraic tools. What 
we are looking at is somewhere at the intersection of algebra and quantum computing.


The idea of representation theory, is to study groups by looking at how they act on vector spaces. This allows us 
to use the tools of linear algebra to study groups. In particular, under the right conditions, we will see that 
quantum oracles naturally give us unitary representations of their underlying groups.

This chapter will focus on developing the tools of representation theory so that we may make this connection 
between group theory and quantum oracle problems more precise. I'm really just writing some stuff off the top of my 
head, I actually have no idea where this is going. 



In this thesis, we will assume that all representations are over $\C $. 
\begin{definition}
    A \emph{representation} of a group $G$ over $\C$ is a homomorphism $\rho$ from $G$ to $\GL(n,F)$, for some $n$. 
    The degree of $\rho$ is the integer $n$.

    We say that $\rho$ is a \emph{faithful} representation if $G \cong \rho(G)$.
\end{definition}


In linear algebra, when given a linear map, we commonly looked for the subspaces it was invariant on. These spaces 
were so important that we gave them their own name, \emph{eigenspaces} 



\begin{definition}
    Irreducible representation
\end{definition}


\begin{definition}
    Equivalent representations? I don't know if I need this.
\end{definition}


\begin{theorem}
    Let $V$ and $W$ be irreducible \CG-modules.

    \begin{enumerate}
        \item If $\varphi : V \rightarrow W$ is a \CG-homomorphism, then either $\varphi$ is a \CG-isomorphism, or 
            $v\varphi = 0$ for all $v \in V$. 
            
        \item If $\varphi \rightarrow V$ is a \CG-isomorphism, then $\varphi$ is a scalar multiple of the identity 
            endomorphism $1_v$.
    \end{enumerate}
\end{theorem}

\begin{proof}
    I'll almost certainly include this one.
\end{proof}


\section{Character Theory}

One of the most surprising tools in representation theory is \emph{characters}. It's a pain to have to associate an 
entire matrix to each element of our group. What would happen if we just wrote down the trace of each matrix 
instead?




\begin{definition}
    Let $G$ be a group and $\rho : G \rightarrow \GL(n, \C)$ a representation of $G$. 

    The \emph{character of }$\rho$ is the function $\chi : G \rightarrow \C$ defined for all $g \in G$ by
    \[
        \chi(g) = \text{Tr}\,\rho(G)
  \]
\end{definition}

Recall from linear algebra that the trace can be equivalently defined as the sum of the eigenvalues of a matrix. 
This result makes it clear that the trace, and thus characters, are invariant under change of basis. This is the 
key part of the remarkable properties that the trace seems to have, one might think of it as, the simplest 
invariant of linear maps (depending on how simple you find its cousin, the determinant to be).


Much in the same way that we think of the relationship between matrices and linear maps to be defined `up to change 
of basis', we might think that groups themselves are also only defined up to inner automorphism.

Although character theory is just a shadow of the representation theory above it, the characters of a group carry a 
surprising amount of information about it. In fact, understanding the character theory of a group, is one of the 
best ways to study the representation theory of a group.

All of this prose was just written in like 30 seconds. I have no actual idea if any of this is readable or at all 
true. Anyways, back to the math.























