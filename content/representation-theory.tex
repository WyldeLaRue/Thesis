
\chapter{Representation Theory of Finite Groups}

\section{Introduction}


%Representation theory is a very powerful tool for analyzing groups. Representation theoretic tools give much more 
%profound insight into much of group theory that can be hard to find without it. The idea of representation theory, 
%is to study groups by looking at how they act on vector spaces. This allows us to use the tools of linear algebra 
%to study groups. In particular, under the right conditions, we will see that quantum oracles naturally give us 
%unitary representations of their underlying groups.

The purpose of this chapter will be to introduce the terminology and results from representation theory that we 
will need to investigate symmetric oracle problems in chapters 3 and 4. A reader familiar with the basic results of 
the representation theory of finite groups can skip ahead to chapter 3. 

This treatment of representation theory largely follows \cite{James&Liebeck}. I have omitted many proofs here and 
instead leave the reader to refer to \cite{James&Liebeck}

Although there are many analogues of the results given here for fields other than $\C$ and for infinite groups, we 
will not express them in their full generality. The representations that will arise in our study of quantum oracle 
problems will be of finite groups and over $\C$. Moreover, many proofs in representation theory are substantially 
simpler and more elegant when the underlying field is restricted to $\C$ since it is algebraically closed. This 
means that $\C$ is often the prototypical field to study representations in.

\section{Representations}

Loosely speaking, a representation of a group $G$ is an encoding of $G$ as a set of linear maps on some vector 
space $V$.  When a basis of $V$ is specified, this yields an explicit representation of each $g \in G$ as a matrix.  
Looking at the possible ways of representing $G$ as a matrix group uncovers much information about the structure of 
$G$ that may have been hidden otherwise. In particular, any encoding of $G$ into a set of matrices will be unique 
only up to a change of basis. This exposes a connection between the symmetry arising in vector spaces by change of 
basis and symmetry inherent in the structure of groups, arising from conjugation.


\begin{definition}
    Let $G$ be a group and $V$ be a vector space over a field $F$. A \emph{representation} of $G$ is a homomorphism 
    $\rho : G \rightarrow \Aut(V)$, where $\Aut(V)$ denotes the group of invertible linear maps from $V$ to itself.  
    Although, the representation is defined by the homomorphism $\rho$, it is common instead to refer to the 
    representation by the vector space $V$ it is acting on.
    
    Equivalently, a representation can be characterized as an action of a group $G$ on a vector space $V$. In this 
    case, the action is required to be compatible with the vector space structure on $V$. That is, in addition to 
    the requirements for an action of a group on a set, we must have
\[g \cdot (\alpha v_1 + v_2) = g\cdot \alpha(v_1) + g\cdot(v_2)\]
    for all $g \in G,\ v_1, v_2 \in V,$ and $\alpha \in F$.
    We denote the action of $g \in G$ on a vector $v$ as $g \cdot v$ or $gv$.
\end{definition}


    There is yet a third way to express the information of a representation. Although this is the most abstract, it 
    provides the most concrete language for working with representations.
    
    First we must define an object called the \emph{group algebra} that is naturally associated to every group.
    One can view the group algebra as an encoding of a group into the structure of polynomial arithmetic.


\begin{definition}
    Let $G$ be a finite group of order $n$. The \emph{group algebra} of $G$, denoted $\C[G]$ or \CG, is defined as 
    the vector space of $\C$ with one basis element for every $g \in G$. Thus every element of $\C[G]$ is of the 
    form
    \[
        \alpha_1 g_1 + \alpha_2 g_2 + ... + \alpha_n g_n
    \]
    where $\alpha_1, ..., \alpha_n \in \C$ and $\{g_1, ..., g_n\} = G$.
    
    The operations of addition, and scalar-multiplication are inherited from the properties of vector spaces.  
    except we define the multiplication of monomials to follow the group structure.
    For any $\alpha_1, \alpha_2 \in \C$ and $g_1, g_2 \in G$ we define
    \[
        (\alpha_1g_1)(\alpha_2g_2) = (\alpha_1\alpha_2)(g_1g_2)
    \]
    where $\alpha_1\alpha_2$ is the product of $\alpha_1, \alpha_2$ in $\C$ and $g_1 g_2$ is the product of $g_1, 
    g_2$ in $G$. In particular, this means $\C[G]$ is a commutative ring if and only if $G$ is abelian.
\end{definition}

The connection between the group algebra and representations is made clear using the theory of modules. For a 
comprehensive treatment of modules, refer to Dummit and Foote \cite{foote} chapter 10.

\begin{definition}
    Let $R$ be a ring. An $R$-module is a pair $(M, \cdot)$ where $M$ is an abelian group and an action of $R$ 
    on $M$, that is a map $R \times M \rightarrow M$ that is associative and distributes over addition in $R$.

    One might think of an $R$-module as the ring analogy to a $G$-set. Although the action is in some sense the 
    most defining part of an $R$-module, the object $M$ being acted on is referred to as the $R$-module
 \end{definition}

A representation $\rho : G \rightarrow \Aut(V)$ defines a $\C G$-module structure on $V$ by the action
\[
    (\alpha_1g_1 + ... + \alpha_ng_n) v := \alpha_1\rho(g_1)v + ...+ \alpha_n\rho(g_n)v
\]
Conversely, a $\C G$-module with action $\varphi : \C G \times V \rightarrow V$ determines a representation $\rho$ 
defined by $\rho(g) = \varphi(1g)$. This establishes a correspondence between representations of $G$ and 
\CG-modules.


\section{The Structure of Representations}

In linear algebra one studies linear maps looking at the subspaces that are invariant under them. The big result in 
linear algebra is that a linear map is determined entirely by its invariant subspaces. The methods for studying 
representations should feel analogous to the methods used for studying linear maps. 

Instead of considering the subspaces invariant under just one linear map, we want the subspaces that are invariant 
under the action of the entire group. 

\begin{definition}
    Let $\rho$ be a representation of a group $G$ on a vector space $V$. A \emph{subrepresentation} is a vector 
    subspace $W \subset V$ that is invariant under the action of $G$. That is, $g\cdot w \in W$ for all $g \in G$, 
    $v \in W$.
 

    A representation is said to be irreducible if it has no nontrivial subrepresentations (the only $G$-invariant 
    subspaces of $V$ are $\{ 0 \}$ and $V$ itself).
\end{definition}

\begin{theorem}[Maschke's Theorem]
    Let $G$ be a finite group and $V$ a representation of $G$ over $\C$. If $W$ is a subrepresentation of $V$, then 
there is a subrepresentation $W'$ such that $V = W \oplus W'$. (Here and throughout this chapter, we use $\oplus$ 
to denote the direct sum of vector spaces).
\end{theorem}


\begin{proof}
    We are given that $W$ is a subrepresentation of $V$. Let $\pi : V \rightarrow W$ be a projection of $V$ onto 
    $W$. Although $\ker \pi \oplus W = V$, we do not know that $\ker \pi$ must be a subrepresentation. In fact, it 
    will not be a subrepresentation in general. We will need to modify it slightly.

    We construct a new function $\pi_G : V \rightarrow W$ by averaging over $G$. Define it by
    \[
        \pi_G(v) = \frac{1}{|G|}\sum_{g \in G} g \cdot \pi(g \cdot v) \]
    Now let $h \in G$ and $u \in V$.
    \[
        h \cdot \pi_G(u) = \frac{1}{|G|} \sum_{g \in G} hg \cdot \pi(g \cdot v)
    \]
    We can reindex our sum to get
    \[
        h \cdot \pi_G (u) = \frac{1}{|H|} \sum_{g \in G} g \cdot \pi(h^{-1}g \cdot v)
    \]
    
\end{proof}



\begin{theorem}
    Every representation over $\C$ decomposes into irreducible subrepresentations. i.e.
    If $(\rho, V)$ is a representation of group $G$ over $\C$, then there exist irreducible subrepresentations 
    $W_1, ...  W_n$ such that $V = W_1 \oplus W_2 \oplus ... \oplus W_n$.
\end{theorem}

\begin{proof}
    If $V$ is irreducible, then we are done. Otherwise, it must have some subrepresentation $W$. By Maschke's 
    Theorem, there is a subrepresentation $U$ such that $V = U \oplus W$. Now, if $U$ and $W$ are irreducible, we 
    are done. Otherwise, they admit subrepresentations by Maschke's Theorem. We repeat this process inductively 
    until each subrepresentation is irreducible. This is guaranteed to terminate in a finite number of iterations 
    because $\dim(V)$ is finite.
\end{proof}
This result establishes the basic philosophy of representation theory: the set of irreducible representations of a 
group tells us almost everything about the group.  Indeed, the representation theory of finite groups was one of 
the key developments that lead to the classification of finite simple groups.  Much of the rest of this chapter 
will be devoted to developing tools that help us find the set of irreducible representations. However, there is one 
caveat though, we do not as of yet know this decomposition into irreducible subrepresentations is unique. With the 
addition of a bit more machinery, we will show that this is indeed the case.
 
%\begin{theorem}[Schur's Lemma]
    %If M and N are two finite-dimensional irreducible representations of a group $G$, and $\varphi:M \to N$ is a 
    %linear transformation that commutes with the action of the group, then $\varphi$ is either an isomorphism or 
    %the $0$ map.
%\end{theorem}

%\begin{proof}
    %See \cite{steinberg} chapter 4.
%\end{proof}
%\begin{theorem}[Schur's Lemma]
    %Let $V$ and $W$ be irreducible \CG-modules.

    %\begin{enumerate}
        %\item If $\varphi : V \rightarrow W$ is a \CG-homomorphism, then either $\varphi$ is a \CG-isomorphism, or 
            %$v\varphi = 0$ for all $v \in V$. 
            
        %\item If $\varphi V \rightarrow V$ is a \CG-isomorphism, then $\varphi$ is a scalar multiple of the 
            %identity endomorphism $1_v$.
    %\end{enumerate}
%\end{theorem}

%\begin{proof}
    %I'll almost certainly include this one.
%\end{proof}


\section{Character Theory}


%Recall from linear algebra that the trace can be equivalently defined as the sum of the eigenvalues of a matrix.  
%This result makes it clear that the trace, and thus characters, are invariant under change of basis. This is the 
%key part of the remarkable properties that the trace seems to have, one might think of it as, the simplest 
%invariant of linear maps (depending on how simple you find its cousin, the determinant to be).


%Much in the same way that we think of the relationship between matrices and linear maps to be defined `up to change 
%of basis', we might think that groups themselves are also only defined up to inner automorphism.

%Although character theory is just a shadow of the representation theory above it, the characters of a group carry a 
%surprising amount of information about it. In fact, understanding the character theory of a group, is one of the 
%best ways to study the representation theory of a group.

%All of this prose was just written in like 30 seconds. I have no actual idea if any of this is readable or at all 
%true. Anyways, back to the math.

\def\Tr{\text{Tr}\,}

\begin{definition}
    Let $M$ be an $n \times n$ matrix. The \emph{trace} of $M$ is defined to be the sum of the diagonal entries of 
    $M$ i.e.
    \[
        \Tr(M) := \sum_{i=1}^n M_{ii}
    \]
    Alternatively, some authors define the trace to be the sum of the eigenvalues of $M$. As it turns out, these 
    two values are always equivalent. This result can be seen as a consequence of the Jordan canonical form.
\end{definition}

\begin{definition}
    Let $\rho : G \rightarrow V$ be a representation. The map $\chi : G \rightarrow \C$ given by $\chi(g) = 
    \text{Tr}\,\rho(g)$ is called the \emph{character} of the representation $V$.
\end{definition}

At first glance, it doesn't seem like characters should be very useful at all. A character only associates a single 
complex number to each $g \in G$ while a representation associates an entire matrix of complex numbers. Why are we 
throwing all of this information away?

Surprisingly, the characters encode almost all of the information we need about the group. This is maybe the dream 
of every student taking their first linear algebra course. Many students complain about having to row reduce or 
write out determinants for $4 \times 4$ matrices, but very few complain about at being asked to add up the four 
numbers on the diagonal.  Many of the results on characters seem too good to be true.

Although it seems crazy at first to take the trace and throw everything else away, the fact that this works isn't 
actually so surprising upon further inspection.

%\begin{definition}
    %A \emph{class function} of a group $G$ is a function $\varphi : G \rightarrow \C$ that is constant on the 
    %conjugacy classes of $G$. As we have already seen, the characters of a group are always class functions.
    
    %The set of class functions on a group $G$ form a vector space under pointwise addition. This space has an 
    %inner-product given by
    %\[
        %\langle \varphi, \psi \rangle := \frac{1}{|G|} \sum_{g \in G} \varphi(g)\overline{\psi(g)}
    %\]
%\end{definition}

%The trace can be thought of in some sense as the simplest map from $\GL(n, \C) \rightarrow \C$ that is invariant 
%under change of basis.

%\begin{theorem}
    %The characters of a group $G$ are constant on the conjugacy classes of $G$. That is, for all $g,h \in G$, 
    %$\varphi(g) = \varphi(hgh^{-1})$
%\end{theorem}

%\begin{proof}
    %Let $\chi$ be a character of $G$ arising from the representation $\rho$. Let $g, h \in G$. Recall from linear 
    %algebra that the trace of a matrix is equivalent to the sum of its eigenvalues.  Thus the trace is invariant 
    %under change of basis. Therefore
    %\[
        %\chi(hgh^{1}) = \Tr \rho(hgh^{-1}) = \Tr \rho(h) \rho(g)\rho(h^{-1}) = \Tr \rho(g) = \chi(g)
    %\]
%\end{proof}


I

\begin{definition}
    A \emph{class function} of a group $G$ is a function $\varphi : G \rightarrow \C$ that is constant on the 
    conjugacy classes of $G$. That is, for all $g,h \in G$, $\varphi(g) = \varphi(hgh^{-1})$.
    
    The set of class functions on a group $G$ form a vector space under pointwise addition. This space has an 
    inner-product given by
    \[
        \langle \varphi, \psi \rangle := \frac{1}{|G|} \sum_{g \in G} \varphi(g)\overline{\psi(g)}
    \]
\end{definition}


Is is often convenient to write explicit class functions as tables of the form 

\begin{center}
\begin{tabular}{c|ccccc}
    $g$ & $g_1$ & $g_2$ & $\cdots$ & $g_n$  \\ \hline
    $\varphi(g)$ & $\varphi(g_1)$ & $\varphi(g_2)$ & $\cdots$ & $\varphi(g_n)$
\end{tabular}
\end{center}
where $g_1, ... g_n$ are representatives from each of the $n$ distinct conjugacy classes of $G$.




\begin{theorem}
    The characters of a group $G$ are class functions from $G \rightarrow \C$.
\end{theorem}

\begin{proof}
    Let $\chi$ be a character of $G$ arising from the representation $\rho$. Let $g, h \in G$. Recall from linear 
    algebra that the trace of a matrix is equivalent to the sum of its eigenvalues.  Thus the trace is invariant 
    under change of basis. Therefore
    \[
        \chi(hgh^{1}) = \Tr \rho(hgh^{-1}) = \Tr \rho(h) \rho(g)\rho(h^{-1}) = \Tr \rho(g) = \chi(g)
    \]
\end{proof}

This result reveals some of the intuition behind why the trace becomes so useful. The trace can be thought of in 
some sense as the simplest map from $\GL(n, \C) \rightarrow \C$ that is invariant under change of basis. 

%\begin{example}
    %Let $\rho : D_8 \rightarrow \GL(n, \C)$ be the representation defined in example 1. The associated character 
    %$\chi_\rho$ is given by
    %\TODO{!}
%\end{example}


%\begin{proposition}
    %Let $\chi$ be a character of a group $G$. Suppose $g \in G$ and $g$ has order $m$. Then
    %\begin{enumerate}
        %\item $\chi(1) = \dim V$.
        %\item $\chi(g)$ is a sum of $m$th roots of unity.
        %\item $\chi(g^{-1}) = \overline{\chi(g)}$.
        %\item $\chi(g)$ is a real number if $g$ is conjugate  to $g^{-1}$
    %\end{enumerate}
%\end{proposition}

%\begin{proof}
    %By definition of a character, $\chi(g)$ is given by $\Tr \rho(g)$ for some representation $\rho : G \rightarrow 
    %\GL(n, \C)$.
    %(1) $\chi(1) = \Tr I$

    %\TODO{Finish or remove this}
%\end{proof}




\subsection{Irreducible Characters}

\begin{definition}
    Let $G$ be a group. We say that $\chi$ is a character of $G$ if it is the character of some representation of 
    $G$.

    Moreover, if $\chi$ is the character of an irreducible representation of $G$, we say that $\chi$ is an
    \emph{irreducible character of} $G$.
\end{definition}

The irreducible characters are particularly special class functions.
\begin{theorem}
    The set of distinct irreducible characters of a group $G$ form an orthonormal basis for the space of class 
    functions on $G$.
\end{theorem}

\begin{theorem}
    Let $V$ be a representation of a group $G$ and let $\psi$ be its associated character.  $V$ is irreducible if 
    and only if and only if $\langle \psi, \psi \rangle = 1$ .
\end{theorem}

\begin{proof}
    See \cite{James&Liebeck} Chapter 14
\end{proof}




\begin{corollary}
    The number of irreducible characters of $G$ is equal to the number of conjugacy classes of $G$.
\end{corollary}
\begin{proof}
    Let $C_1, C_2, ..., C_n$ be the conjugacy classes of $G$. For each $1 \leq i \leq n$ define $\varphi : G 
    \rightarrow \C$ by
    \[
        \varphi_i(G) =
        \begin{cases}
            1 &\text{if } g \in C_i \\
            0 &\text{if } g \not\in C_i
        \end{cases}
    \]
    The set $\{\varphi_1, ..., \varphi_n \}$ forms an orthonormal basis for the space of class functions from $G$ 
    to $\C$. In fact, one might consider this to be the standard basis on this space. By the previous theorem, the 
    set of irreducible characters also form a basis for the space of class functions, so they must be the same 
    size.
\end{proof}

    
\begin{corollary}
    Let $\chi_1, ..., \chi_n$ be the irreducible characters of a group $G$. Let $\varphi : G \rightarrow \C$ be a 
    class function on $G$. Then $\varphi$ decomposes as 
    \[
       \varphi = \alpha_1 \chi_1 + ... + \alpha_n \chi_n
    \]
    where $\alpha_i = \langle \varphi, \chi_i \rangle$.
\end{corollary}
\begin{proof}
    This follows straightforwardly from the properties of inner-products.
\end{proof}

\begin{lemma}
    Let $(\varphi, V)$ be a representation of group $G$ with subrepresentations $U$ and $W$ such that $V = U \oplus 
    W$. Let $\chi_V, \chi_U, \chi_W$ be the characters of $V, U,$ and $W$ respectively.

    Then $\chi_V = \chi_U + \chi_W$.
\end{lemma}

\begin{theorem}
    In the special case that $\varphi$ is a character, then its decomposition into irreducibles given by
    \[
        \varphi = m_1\chi_1 + m_2 \chi_2 + ... + m_n \chi_n
    \]
    is guaranteed to have non-negative integral coefficients. That is, each $m_1,...,m_n$ must be a nonnegative 
    integer. These integers count the multiplicity of the associated irreducible subrepresentations appearing in 
    the decomposition of the associated representation of $\psi$.
\end{theorem}

\begin{proof}
    Let $(\rho, V)$ be a representation of $G$ and $\chi$ the character of $\rho$. By Theorem 2.3, there exist 
    irreducibles $W_1, ..., W_n$ such that $V = W_1 \oplus \cdots \oplus W_n$. Let $\psi_1, ..., \psi_n$ denote the 
    characters of $W_1, ..., W_n$ respectively. Note that these characters need not necessarily be distinct.

Let $\psi_{k_i}$ be the set of unique characters in $\psi_1, ..., \psi_n$. This must be the set of irreducible 
characters, therefore we know that $\chi$ decomposes as $\chi = \alpha_1\psi_{k_1} + ... + \alpha_n\psi_{k_n}$.  
However, by the previous lemma, we also have $\chi = \psi_1 + ... +  \psi_n$. Therefore $\alpha_i$ must be exactly 
the multiplicity that the character $\psi_{k_i}$ appears in $\psi_1, ..., \psi_m$.
\end{proof}

\begin{proposition}
    Let $\chi$ be the character of a representation $(\rho, V)$ of group $G$. As is common in representation theory 
    literature, let $1$ denote the identity of $G$. Then
    \[
        \chi(1) = \dim(V)
    \]
    The value of $\chi(1)$ is referred to as the degree of the $\chi$.
\end{proposition}

\begin{proof}
    By definition $\chi(1) = \Tr(\rho(1)) = \Tr(I_V) =  \dim(V)$.
\end{proof}

\begin{theorem}
    Let $\chi_1, ... ,\chi_n$ be the irreducible characters of $G$. Then

    \[
        |G| = \chi_1(1)^2 + ... + \chi_n(n)^2
    \]
\end{theorem}

\begin{definition}
    The characters of degree $1$ are referred to as the \emph{linear characters}. They can equivalently be thought 
    of as the homomorphisms from $G$ into $\C\setminus \{0\}$.
\end{definition}

    
    Writing the information of the characters of a group into a table, gives us an extremely compact way write down 
    the important information associated to a group.

\begin{definition}
    A character table of a group $G$ is a table with entries 
\end{definition}

\subsection{Products of Characters}

These results will be about the tensor product of representations. Just as in Chapter 1, giving a rigorous 
definition of the tensor product is probably outside the scope of this thesis. A good treatment of the tensor 
product can be found in chapter 10 of Dummit and Foote \cite{foote} and a treatment of what is relevant for 
representation theory can be found in chapter 19 of \cite{James&Liebeck}.

\begin{definition}
    Let $G$ be a group. Define a product on the space of class functions of $G$ by pointwise multiplication. i.e.
    for all class functions $\chi, \psi$ of $G$, 
    \[
        \chi\psi(g) = \chi(g)\psi(g) \text{ for all } g \in G
    \]
    This is sometimes referred to as the \emph{tensor product} of characters.
\end{definition}

If one uses purely character theoretic arguments, the tensor product is actually not needed. In fact, Steinberg's 
text \cite{steinberg} proves the major results in the representation theory of finite groups without invoking the 
tensor product. We take the same approach, however as we need to connect product characters to tensor products of 
quantum state spaces, we will have to take the following result as given.

\begin{proposition}
    Let $G$ be a group. Let $V$ and $W$ be representations of with characters $\chi$ and $\psi$ respectively. Then 
    the character of the representation $V \otimes W$ is the product character $\chi\psi$
\end{proposition}


\begin{theorem}\label{thm:Burnside_Brauer}(Burnside-Brauer)
    Let $\chi$ be a faithful character of $G$, and suppose that $\chi(g)$ takes precisely $r$ different values as 
    $g$ varies over all the elements of $G$. Then every irreducible character of $G$ is a constituent of one of the 
    powers $\chi^0, \chi^1, ... , \chi^{r-1}$
\end{theorem}
See James and Liebeck \cite{James&Liebeck}, chapter 19.

