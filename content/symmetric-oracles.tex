
\chapter{Symmetric Oracle Problems}



In this chapter, we introduce a specific class of oracle problems in which the hidden information is sampled from a 
group $G$. We then summarize the results of a recent paper \cite{jamie} of Pommersheim and Copeland that show the 
optimal query complexity of such problems is determined entirely by the character theory of $G$. 





\section{Symmetric Oracle Problems}
This follows the approach taken in \cite{jamie}.


\subsection{Classical learning}

Let $G$ be a finite group that acts on a finite set $\Omega$. Denote the action of $g \in G$ on $x \in \Omega$ as 
$g \cdot x$. Suppose an element $g \in G$ is selected uniformly at random and hidden. A learner is given oracle 
access to the action of $g$ as the function $f : \Omega \rightarrow \Omega$ given by $f(x) = g \cdot x$. That is, 
the learner can input elements of $\Omega$ as queries and see how $g$ acts on them. The learner's goal is to 
determine $g$ in as few queries as possible.

\begin{example}
    Consider the example of the action of the symmetric group $S_n$ on $\{1, ..., n\}$:
    Suppose a permutation $\sigma$ is chosen uniformly at random from the set of permutations on $\{1, ..., n\}$ 
    and hidden. We are given access to an oracle that takes an $x \in \{1,...,n\}$ as input and returns 
    $\sigma(x)$. How many queries do we need to determine $\sigma$?
\end{example}

This task requires $n-1$ queries in the worst case. To see this, suppose $\sigma$ were chosen to be the identity 
permutation.  Even after seeing that $\sigma$ fixes all but $2$ elements of $\{1,...,n\}$, we still cannot 
distinguish whether $\sigma$ is the identity or the transposition of the last two elements.



\bigskip

This problem turns out historically to have been important to computational group theory and the efficient 
implementation of permutation groups in computer algebra systems. Storing groups in memory as permutations is very 
cumbersome and so instead one expresses the group by its action on a \emph{base}, which is defined as a sequence
\[
    B = [\beta_1, \beta_2, ... \beta_k]
\]
of $k$ distinct elements for which the only $g\in G$ that fixes every $\beta_i \in B$ is the identity. The size of 
the smallest base of a permutation group is exactly the worst-case number of queries that would be needed to 
identify a permutation in this group with oracle access to its action. A very good resource on this is the text 
\cite{seress} of Seress. 


\subsection{Quantum Learning}

This problem naturally extends to the quantum setting. Let $\pi : G \rightarrow \Aut(\Omega)$ an action of $G$ on a 
set $\Omega$. Let $V$ be a vector space with orthornormal basis $\Omega$. Then the action $\pi : G \rightarrow 
\Aut(\Omega)$ defines a permutation of basis vectors and thus linearly extends to a unitary representation 
$\rho_\pi : G \rightarrow U(V)$. Here $U(V)$ denotes the group of unitary transformations of $V$.

This construction of a representation from a group action on a set is common in representation theory. It is common 
to refer to the character associated to this representation as the \emph{permutation character} often denoted 
$\chi_\pi$ or just $\pi$. Since $\rho_\pi$ assigns a permutation matrix to each $g$ in $G$, the trace of 
$\rho_pi(g)$ for any $g \in G$ is exactly the number of points that $g$ fixes under the action. 

\begin{definition}
    The permutation character $\chi_\pi$ of an action $\pi: G \rightarrow \Aut(\Omega)$, commonly just written 
    $\pi$, is given by
    \[
        \chi_\pi(g) := \text{Fix}(g)
    \]
    where $\text{Fix}(g)$ is the number of elements in $\Omega$ fixed by the action of $G$.
\end{definition}

\subsection{Query complexity of quantum symmetric learning}
Finding the classical query complexity of such a problem is relatively straightforward and there are many 
well-known algorithms for determining it. However, in the quantum case, it is less obvious how to determine the 
optimal number of queries.

A recent result of Pommersheim and Copeland \cite{jamie} reduces this problem entirely to a problem of 
representation theory. We state theorem 4.2 from their paper
    
\begin{theorem}
    Suppose $G$ is a finite group and $\pi : G \rightarrow U(V)$ is a unitary representation of $G$. Then an 
    optimal $t$-query algorithm to solve symmetric oracle discrimination succeeds with probability
    \[
        P_{opt} = \frac{d_v}{|G|}
    \]
    where
    \[
        d_v = \sum_{\chi \in I(V^{\otimes t})} \chi(e)^2
    \]
    where $I(V)$ denotes the set of irreducible characters of $V$.
\end{theorem}

\begin{proof}
    For the proof of this fact, we refer to the paper itself.
\end{proof}


As mentioned before, the optimal number of queries to solve this problem classically for any permutation group is 
known as the \emph{base size}. Since every permutation group also has an analogous learning problem in the quantum 
setting, we might call minimal number of queries needed for exact learning the \emph{quantum base size} and the 
number of queries needed for learning with sufficiently high probability the \emph{bounded quantum base size}.
Using Theorem 3.1 we can define these terms in entirely in group theoretic language.
\begin{definition}
    Let $G$ be a finite group acting on a finite set $\Omega$. Let $\pi$ be the permutation character of this 
    action. The \emph{quantum base size of} $G$, denoted $\gamma (G)$, is the smallest $t$ such that every 
    irreducible character of appears in $\pi^{\otimes t}$.

    The \emph{bounded quantum base size of } $G$, denoted $\gamma_\text{bdd}(G)$, is the smallest $t$ such that
    \[
        \sum_{\chi \in I(V^{\otimes t})} \chi(e)^2 \geq \frac{2}{3}
    \]

\end{definition}











%We state Theorem 2.1 from the paper
%\begin{theorem}
%Suppose $(G,V,f)$ describes an instance of coset identification. Then there exists a $t$-query quantum algorithm to 
%determine $f(a)$ if and only if there exists a $t$-query nonadaptive algorithm which does the same.
%\end{theorem}




















