\documentclass[12pt,twoside]{reedthesis}
\usepackage{graphicx,latexsym} 
\usepackage{amssymb,amsthm,amsmath}
\usepackage{longtable,booktabs,setspace} 
\usepackage[hyphens]{url}
\usepackage{rotating}
\usepackage{natbib}
% Comment out the natbib line above and uncomment the following two lines to use the new 
% biblatex-chicago style, for Chicago A. Also make some changes at the end where the 
% bibliography is included. 
%\usepackage{biblatex-chicago}
%\bibliography{thesis}

\usepackage{graphicx}
% \usepackage{times} % other fonts are available like times, bookman, charter, palatino

\title{My Final College Paper}
\author{Your R. Name}
% The month and year that you submit your FINAL draft TO THE LIBRARY (May or December)
\date{May 200x}
\division{Mathematics and Natural Sciences}
\advisor{Advisor F. Name}
%If you have two advisors for some reason, you can use the following
%\altadvisor{Your Other Advisor}
%%% Remember to use the correct department!
\department{Mathematics}

\setlength{\parskip}{0pt}
%%
%% End Preamble
%%
%% The fun begins:
	 

\setcounter{chapter}{2}


\begin{document}
\chapter{Quantum Base Size of $GL(2,q)$}


\section*{Character Theory of GL(2,q)}


We will take the following things for granted:

\begin{itemize}
\item For every $q=p^n$ there exists exactly one field up to isomorphism.  We will call that field $\mathbb{F}_q$


\item For every $s\in\mathbb{F}_q$, the sum of $s$ with itself $p$ times is 0.  i.e. $ps=0$.  This is usually stated as  $\mathbb{F}_q$ has characteristic $p$.

\item The group $(\mathbb{F}^*_q,x)$ is cyclic.


\end{itemize}



\subsection*{A Useful Proposition}

Let $F= \mathbb{F}_{q^2}$ 
and 
$S=\{ s \in F | s^q = s \}$

Then

\begin{enumerate}
    \item $S$ is a subfield of $F$ of order $q$ (hence $\mathbb{F}_q \cong S$)
    \item If $r\in{F}$ then $r+r^q, r^{1+q}\in{S}$
\end{enumerate} 

We will use this from here on out to identify the subfield, $S$, as $\mathbb{F}_q$.


\subsection*{Proof of our useful propsition}
\begin{enumerate}

\item Suppose that $s,t,\in{S}$.  Then ${(s+t)}^q = s^q + t^q = s+t$ by (Frobonius Homomorphism / Freshman's Dream.)
\\ Thus $s+t\in{S}$.
\\This gives us that $(S,+)$ is an abelian group (since $1\in{S}$) and since ${(st)}^q = s^qt^q = st$ we get $(S^*,x)$ is also an abelian group.
 
\item Since $(\mathbb{F}^*_{q^a})$ is a group of order $q^2 -1$, it must that $r^{q^2} = r$ for all $r\in{\mathbb{F}_{q^2}}$ by Larange (?)
\\This implies that ${(r+r^q)}^q = r^q + r^{q^2} = r + r^q$ so $r+r^q$ and $r^{1+q} \in S.$

\end{enumerate}

\subsection*{Some Notation}
We introduce some useful notation:
 
Let $\epsilon$ be a generator of the cyclic group $\mathbb{F}^*_{q^2}$ and let $\omega = e^{\frac{2\pi i}{q^2-1}}$. 

Furthermore, suppose $r\in{\mathbb{F}_{q^2}}$.

We may write $r=\epsilon^m$ for some $m$ and let $\bar{r} = \omega^m$.

Then the map $r\mapsto\bar{r}$ is an irreducible character of  $\mathbb{F}^*_{q^2}$.  Moreover, every irreducible character has the form $r\mapsto\bar{r}^j$ for some integer $j$.

Breaking this down further, let $x_j$  be defined by $x(r)=\bar{r}^j$.

Then of course this is a character since it is a homomorphism from an abeliean group into $\mathbb{C}^*.$


\section*{The Size of GL(2,q)}

Remark that we can trivially represent GL(2,q) as the set of matrices of the form

$$ \begin{vmatrix}
a&b\\
c&d\\
\end{vmatrix}
$$

with determinate $\neq{0}$.\\

Thus a matrix

$$ \begin{vmatrix}
a&b\\
c&d\\
\end{vmatrix}
$$

belongs to GL(2,q) if and only if its rows are linearly independent.  Therefore $(a,b)$ can be anything as long as they are not both zero ($q^2 -1$ choices) and then $(c,d)$ can be anything 
that is not a scaler multiple of $(a,b)$ giving us $q^2-q)$ choices.
Therefore GL(2,q) has $(q^2-q)(q^2-q)$ elements.\\
This argument nice generlizes to GL(n,q).

\section*{Conjugacy Classes of GL(2,q)}

There are 4 families of conjugacy classes of G.  3 of these are easy, one is hard.

\begin{enumerate}

\item 
$ \begin{vmatrix}
a&b\\
0&d\\
\end{vmatrix}
$
is conjuagte to
$ \begin{vmatrix}
a'&b'\\
0&d'\\
\end{vmatrix}
$
only if $\{a,c\} = \{a',c'\}$ \\
since conjuagate matrices have the same eigenvalues.


\item The matrices 
\[sI = \begin{vmatrix}
a&b\\
0&d\\
\end{vmatrix}
\]

belongs to the center of G.  They give us $q-1$ (the number of choices for s) conjugacy classes of size one.


\item 
Let
\[ g= \begin{vmatrix}
a&b\\
c&d\\
\end{vmatrix}
\in{G}   \textnormal{ and  }    u_s =
\begin{vmatrix}
s&1\\
0&s\\
\end{vmatrix}
\]

Then
\[
gu_s =
\begin{vmatrix}
as&a+bs\\
cs&c+ds\\
\end{vmatrix}
\]

and

\[
u_sg =
\begin{vmatrix}
as&d+bs\\
cs&ds\\
\end{vmatrix}
\]

so $g$ belongs to the centralizer of $u_s$ if and only if $c=0$ and $a=d$.\\
Thus the matrices $u_s$ ($s\in{\mathbb{F}_2}$) give us $q-1$ conjugacy classes.  The order of the centralizer is $(q-1)q$, so by the Orbit-Stabilizer Theorem, each conjugacy class contains $q^2-1$ elements.


\item
Now let $d_{s,t} =
\begin{vmatrix}
s&0\\
0&t\\
\end{vmatrix}$\\

Note that

$
\begin{vmatrix}
0&1\\
1&0\\
\end{vmatrix}^{-1}
$
$
\begin{vmatrix}
s&0\\
0&t\\
\end{vmatrix}
$
$
\begin{vmatrix}
0&1\\
1&0\\
\end{vmatrix}
$
=
$
\begin{vmatrix}
t&0\\
0&s\\
\end{vmatrix}
$\\

On the other hand, if $\neq{t}$, then we have $gd_{s,t} = d_{s,t}g$ if and only if $b=c=0.$  Thus, the matrices $d_{s,t}$ ($s,t, \in{\mathbb{F}^*_q}, s\neq{t}$) 
give us $/frac{(q-1)(q-2)}{2}$ conjuagacy classes.  The centralizer order is $(q-1)^2$, so again by the orbit-stabalizer theorem each conjugacy class contains $q(q+1)$
elements. 




\item
Finally, consider
$$
v_{r} =
\left[
\begin{matrix}
    0 & 1 \\
    -r^{1+q} & r + r^2
\end{matrix}
\right]
\left(
r \in  \mathbb{F}_{q^2}\setminus \mathbb{F}_q
\right)
$$
By our initial proposition $v_r \in G$
\newline
The characteristic polynomial of 
$v_r$
is
$$
det
(xI - v_r) =
x (x - (r + r^{2}))
+
r^{1 + q}
= (x-r)(x - r^{2})
$$
so $v_{r}$ has eigenvalues of r and $r^{2}$.
\newline
Since $r \notin  \mathbb{F}_{2}$
we see that $v_{r}$ lies in none of the conjegacy classes we have constructed so far.
Now
$$
gv_{r}= 
\left[
\begin{matrix}
-br^{1+q} & a+b(r+r^{2})\\
-dr^{1+q} & c+d(r+r^{q})
\end{matrix}
\right]
$$ 
and
$$
v_{r}g= 
\left[
\begin{matrix}
c  & d\\
-ar^{1+q} +c(r+r^q) & -br^{1+q} + d(r+r^{q})
\end{matrix}
\right]
$$
Hence
$gv_{r} = v_{r}g$ only if $c = -br^{1+q}$ and $d = a+b(r+r^{2})$
If these conditions hold, then 
$ad-bc = a^{2}+ab(r+r^{2}) + b^{2}r^{1+q} = (a+br) (a+br^{2})$
\newline
Since 
$(a,b) \ne (0,0)$ and $r, r^{q} \notin \mathbb{F}_{q}$ 
we see that $a+br$ and $a+br^{q}$ are non zero.
\newline
Therefore $g\in C_{G}(v_{r}) \iff g = 
\left[
\begin{matrix}
 a & b\\
 -br^{1+q} & a+b(r+r^2)
\end{matrix}
\right]$
\newline 
Thus $|C_G(v_r)| = q^2-1$ and the conjegacy class containing $v_r$ has size $q^2-1$.

\end{enumerate}
\newpage

The matrix $v_t$ has eigenvalues $t$ and $t^q$ so it is not conjugate to $v_r$ unless $t=r$ or $t=r^q$.  
Therefore we can partitian $\mathbb{F}_{q^2}\setminus \mathbb{F}_q$ into subsets of $\{r,r^q\}$.
Each subset gives us a conjugacy class represenative $v_r$ and different subsets give us representatives of different conjugacy classes of G, in fact all of the classes of G.

\subsection*{Conjugacy Classes}
Propostion: There are $q^2-1$ conjugacy classes of GL(2,q) and they are described as follows:

\begin{table}[h]
\begin{tabular}{l|llll}
       &    &    &     &    \\ \hline
class rep  & $sI$ & $u_s$ & $d_{s,t}$ & $v_r$ \\
$|C_G(g)|$     & $(q^2-1)(q^2-q)$  & $(q-1)q$  & $(q-1)^2$   &  $q^2-1$ \\
number of classes & $q-1$ &  $q-1$  & $\frac{(q-1)(q-2)}{2}$   & $\frac{q^2-q}{2}$ 
\end{tabular}
\end{table}


This can be verified by adding to see they sum to the order of the group.







\end{document}
